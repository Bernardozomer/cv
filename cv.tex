\documentclass[11pt]{article}

\usepackage[sfdefault]{biolinum}
\usepackage[T1]{fontenc}
\usepackage[margin=1in]{geometry}
\usepackage[utf8]{inputenc}
\usepackage{titling}

\begin{document}

\pretitle{\begin{flushleft}\huge\bfseries}
\title{Bernardo Barzotto Zomer}
% Remove the date from the title.
\date{}
\posttitle{\par\end{flushleft}\hrule\vspace{-1in}}
\setlength{\droptitle}{-0.75in}
\maketitle

\noindent
\begin{tabular}{ l@{} c l l@{} c l }
Email		&:&bernardobarzottoz@gmail.com	&Telefone	&:&+55 (51) 99652-7012\\
GitHub		&:&github.com/Bernardozomer		&LinkedIn	&:&linkedin.com/in/bernardo-b-zomer\\
Lattes		&:&lattes.cnpq.br/0916138424420417
\end{tabular}

\section*{Interesses}

Computação de alto desempenho, programação paralela e distribuída, avaliação de
desempenho e otimização, programação de baixo nível, programação funcional,
algoritmos, pesquisa e desenvolvimento, desenvolvimento de software, projeto e
desenvolvimento de jogos, Linux e desenvolvimento de \textit{back end}.

\section*{Formação}

Graduação em \textbf{Ciência da Computação} na Pontifícia Universidade Católica
do Rio Grande do Sul (\textbf{PUCRS}). Início em 2021/1 e previsão de conclusão
em 2024/2.

\section*{Experiência}

\begin{itemize}
	\item \textbf{PUCRS/Dell IT Infra Residency}: Estágio.\\
		Lorem ipsum dolor sit amet, qui minim labore adipisicing minim sint cillum sint consectetur cupidatat.
	\item Grupo de Modelagem de Aplicações Paralelas (\textbf{GMAP}): Bolsa de iniciação científica.\\
		Lorem ipsum dolor sit amet, qui minim labore adipisicing minim sint cillum sint consectetur cupidatat.
\end{itemize}

\section*{Publicações}

Lorem ipsum dolor sit amet, qui minim labore adipisicing minim sint cillum sint consectetur cupidatat.

\section*{Conhecimentos}

\begin{tabular}{ l@{} c l }
	\textbf{Idiomas} &:&Português nativo, Inglês fluente e Espanhol básico;\\
	\textbf{Linguagens} &:&Rust, Python, Shell, Java, C++ e C intermediários; Haskell, C\# e SQL básicos;\\
	\textbf{Bibliotecas e frameworks} &:&OpenMP, OneTBB, Rayon e Crossbeam intermediários; OpenGL básico;\\
	\textbf{Ferramentas} &:&Git, Unity e Godot intermediários; Docker básico;\\
	\textbf{Bancos de dados} &:&Oracle e MongoDB básicos;\\
	\textbf{Nuvem} &:&AWS básico;\\
	\textbf{Sistemas operacionais} &:&Linux e Windows intermediários.\\
\end{tabular}

\section*{Certificados}

\begin{itemize}
	\item Scrum Foundation Professional Certificate, CertiProf, código 79862480.
\end{itemize}

\end{document}
