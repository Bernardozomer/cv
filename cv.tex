\documentclass[11pt]{article}

%% Load babel with the chosen language.
\usepackage[\documentlanguage]{babel}

% Other packages.
\usepackage[sfdefault]{biolinum}
\usepackage{etoolbox}
\usepackage{fontawesome5}
\usepackage[T1]{fontenc}
\usepackage[margin=1in]{geometry}
\usepackage[colorlinks=true, urlcolor=blue, linkcolor=red]{hyperref}
\usepackage[utf8]{inputenc}
\usepackage{titling}
\usepackage{ifthen}

% Miscellaneous document setup.
\setlength{\parindent}{0pt}
\pagenumbering{gobble}

% Generic bilingual command.
\newcommand{\lang}[2]{%
    \ifthenelse{\equal{\documentlanguage}{english}}{#2}{#1}%
    \unskip
}

\robustify{\lang} % Ensures \lang works well in all contexts

% Helper commands.
\newcommand{\cvtitle}[2]{
    \pretitle{\begin{flushleft}}
    \title{\huge \textbf{#1} \hfill \normalsize #2}
    \author{}
    \date{}
    \posttitle{\par\end{flushleft}\hrule\vspace{-0.75in}}
}

\newcommand{\contactinfo}[3]{
    \faIcon{#1} \href{#2://#3}{#3}
}

\newcommand{\otherinfo}[2]{
    \faIcon{#1} {#2}
}

\newcommand{\skillclass}[2]{
    \textbf{#1:} & #2
}

\newcommand{\experience}[2]{
    \subsection*{#1 \hfill \normalfont{#2}}
}

% Document content.
\begin{document}

%% Header: name, job title and contact info.
\cvtitle{Bernardo Barzotto Zomer}{\lang{Engenheiro de Observabilidade}{Observability Engineer}}
\setlength{\droptitle}{-0.75in}
\maketitle

\begin{tabular}{ l l }
    \contactinfo{envelope-square}{mailto}{bernardobarzottoz@gmail.com} &
    \contactinfo{phone-square-alt}{tel}{+55-51-99652-7012} \\
    \contactinfo{linkedin}{https}{linkedin.com/in/bernardo-b-zomer} &
    \contactinfo{github-square}{https}{github.com/Bernardozomer} \\
    \contactinfo{pen-square}{https}{lattes.cnpq.br/0916138424420417} & 
    \otherinfo{map-marker-alt}{Porto Alegre, RS, Brasil} \\
\end{tabular}

%% Body 
\section*{\lang{Competências}{Skills}}

\begin{tabular}{ l l }
    \skillclass{\lang{Ferramentas}{Tools}}{Grafana, Docker, Terraform, Ansible, Git, GitHub;} \\
    \skillclass{\lang{Bancos de dados}{Databases}}{Prometheus, InfluxDB, SQL Server, MySQL, PostgreSQL, MongoDB;} \\
    \skillclass{\lang{Linguagens}{Programming}}{Python, Rust, Java, C, C++, Shell, SQL;} \\
    \skillclass{\lang{Idiomas}{Languages}}
        {\lang{Inglês fluente (apto a leitura, escrita e conversas técnicas), Português nativo.}
        {Fluent in English (reading, writing, technical conversations), native Portuguese.}} \\
\end{tabular}

\section*{\lang{Experiência}{Experience}}

\experience{Dell Technologies}
    {\lang{Engenheiro de Observabilidade}{Observability Engineer}}

\emph{\lang{08/2023 a 02/2025 (Engenheiro de Software I), 02/2025 até o presente (Engenheiro de Software II).}
    {08/2023 to 02/2025 (Software Engineer I), 02/2025 to present (Software Engineer II).}}

\lang{
    Desenvolvimento de dashboards no Grafana para aumentar a proatividade de times de SRE. Correlação de métricas (Prometheus, InfluxDB) com databases SQL. Desenvolvimento de testes sintéticos e alertas. Colaboração com times de engenharia de dados para expandir a cobertura do monitoramento. Desenvolvimento de soluções para incidentes P1 e avaliação das ferramentas oferecidas com base em incidentes passados. Automação de processos de infraestrutura F5 com Python e Flask.
}{
    Development of dashboards in Grafana to boost SRE team proactivity. Correlation of metrics (Prometheus, InfluxDB) with SQL databases. Creation of synthetic tests and alerts. Collaboration with data engineering teams to expand monitoring coverage. Development of solutions for P1 incidents and evaluation of tools based on past incidents. Automation of F5 infrastructure processes with Python and Flask.
}

\experience{\lang{PUCRS/Dell IT Infra Residency}{PUCRS/Dell IT Infra Residency}}
    {\lang{Estágio}{Internship}}

\emph{\lang{04/2023 a 08/2023.}{04/2023 to 08/2023.}}

\lang{
    Residência em infraestrutura de TI através de um treinamento imersivo e capacitação técnica com ênfase no estudo, planejamento e desenvolvimento de processos de automação. Realização de um projeto de IaC com provisionamento de infraestrutura na AWS por meio de Terraform e Ansible, com bancos de dados NoSQL MongoDB com redundância (replicação e sharding) e streaming de dados de vídeo. Monitoramento dos bancos de dados com Grafana e Prometheus.
}{
    IT infrastructure residency through an immersive training program and technical qualification, with an emphasis on studying, planning, and developing automation processes. Developed an Infrastructure as Code (IaC) project by provisioning infrastructure on AWS using Terraform and Ansible, incorporating NoSQL MongoDB databases with redundancy (replication and sharding) and video data streaming. Database monitoring with Grafana and Prometheus.
}

\experience{\lang{Grupo de Modelagem de Aplicações Paralelas (GMAP)}{Parallel Applications Modelling Group (GMAP)}}
    {\lang{Bolsa de Iniciação Científica}{Undergraduate Researcher}}

\emph{\lang{10/2022 a 04/2023.}{10/2022 to 04/2023.}}

\lang{
    Atuação como pesquisador na área de programação paralela e distribuída, com foco em computação de alto desempenho, avaliação de algoritmos, otimização de soluções e ciência de dados. Paralelização do NAS Parallel Benchmarks (NPB) com OpenMP e OneTBB (C++), conversão do kernel EP do NPB para Rust e paralelização de aplicações de streaming.
}{
    Research experience in the field of parallel and distributed programming, focusing on high-performance computing, algorithm evaluation, solution optimization, and data science. Worked on parallelizing the NAS Parallel Benchmarks (NPB) using OpenMP and OneTBB (C++), converting the NPB EP kernel to Rust, and parallelizing streaming applications.
}

\section*{\lang{Certificados}{Certificates}}

\begin{itemize}
	\item \href{https://www.credly.com/badges/69bb404e-ae99-4a93-96b1-228fa98d0859/public_url}
        {AWS Academy Graduate - AWS Academy Introduction to Cloud Semester 1}
\end{itemize}

\section*{\lang{Formação}{Education}}

\lang{
    Graduação em \textbf{Ciência da Computação} na Pontifícia Universidade Católica do Rio Grande do Sul (\textbf{PUCRS}). Início em 2021/1 e previsão de conclusão em 2025/1.
}{
    Bachelor's in \textbf{Computer Science} at Pontifical Catholic University of Rio Grande do Sul (\textbf{PUCRS}). Started in 2021/1, with expected completion in 2025/1.
}

\end{document}
