\documentclass[11pt]{article}

\usepackage[sfdefault]{biolinum}
\usepackage[T1]{fontenc}
\usepackage[margin=1in]{geometry}
\usepackage[colorlinks=true, urlcolor=blue, linkcolor=red]{hyperref}
\usepackage[utf8]{inputenc}
\usepackage{titling}

\setlength{\parindent}{0pt}
\pagenumbering{gobble}

\begin{document}

\pretitle{\begin{flushleft}\huge\bfseries}
\title{Bernardo Barzotto Zomer}
% Remove the date from the title.
\date{}
\posttitle{\par\end{flushleft}\hrule\vspace{-1in}}
\setlength{\droptitle}{-0.75in}
\maketitle

\begin{tabular}{ l l l l }
\textbf{Email:}  & bernardobarzottoz@gmail.com                                             & \textbf{Telefone:} & +55 (51) 99652-7012  \\
\textbf{GitHub:} & \href{https://github.com/Bernardozomer}{Bernardozomer}                  & \textbf{LinkedIn:} & \href{https://linkedin.com/in/bernardo-b-zomer}{in/bernardo-b-zomer}  \\
\textbf{Lattes:} & \href{https://lattes.cnpq.br/0916138424420417}{Bernardo Barzotto Zomer}
\end{tabular}

\section*{Interesses}

Computação de alto desempenho, computação paralela e distribuída, avaliação de
desempenho e otimização, infraestrutura de TI, virtualização, provisionamento,
computação em nuvem, Linux, ciência de dados, programação de baixo nível,
programação funcional, desenvolvimento de \textit{back end}, pesquisa e
desenvolvimento, algoritmos e teoria da computação.

\section*{Formação}

Graduação em \textbf{Ciência da Computação} na Pontifícia Universidade Católica
do Rio Grande do Sul (\textbf{PUCRS}). Início em 2021/1 e previsão de conclusão
em 2024/2.

\section*{Experiência}

\subsection*{PUCRS/Dell IT Infra Residency \hfill \normalfont{Estágio}}

04/2023 até o presente.

Residência em infraestrutura de TI através de um treinamento imersivo e
capacitação técnica com ênfase no estudo, planejamento e desenvolvimento de
processos de automação. Tecnologias e ferramentas utilizadas incluem Linux,
Shell, Docker, serviços AWS, MongoDB, MinIO, Terraform e Grafana.

\subsection*{Grupo de Modelagem de Aplicações Paralelas (GMAP) \hfill \normalfont{Bolsa de Iniciação Científica}}

10/2022 a 04/2023.

Atuação como pesquisador na área de programação paralela e distribuída, com
foco em computação de alto desempenho e, em menor grau, ciência de dados.
Atividades incluem avaliação de algoritmos e otimização e projeto de soluções,
assim como uso de Linux, Shell, C, C++, Rust e bibliotecas de paralelismo como
OpenMP e OneTBB.

\section*{Conhecimentos}

\begin{tabular}{ l l }
	\textbf{Idiomas:} & Português nativo, Inglês fluente e Espanhol básico; \\
	\textbf{Linguagens:} & Rust, Python, Shell, Java, C++ e C intermediários; Haskell, C\# e SQL básicos; \\
	\textbf{Bibliotecas e frameworks:} & OpenMP, OneTBB, Rayon e Crossbeam intermediários; OpenGL básico; \\
	\textbf{Ferramentas:} & Git, Docker, Unity e Godot intermediários; Terraform e Grafana básicos; \\
	\textbf{Armazenamento:} & MongoDB, MinIO, MySQL e Oracle básicos; \\
	\textbf{Sistemas operacionais:} & Linux e Windows intermediários. \\
\end{tabular}

\section*{Certificados}

\begin{itemize}
	\item \href{https://www.credly.com/badges/69bb404e-ae99-4a93-96b1-228fa98d0859/public_url}{AWS Academy Graduate - AWS Academy Introduction to Cloud Semester 1}
\end{itemize}

\end{document}
