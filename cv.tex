\documentclass[11pt]{article}

\usepackage[sfdefault]{biolinum}
\usepackage{fontawesome5}
\usepackage[T1]{fontenc}
\usepackage[margin=1in]{geometry}
\usepackage[colorlinks=true, urlcolor=blue, linkcolor=red]{hyperref}
\usepackage[utf8]{inputenc}
\usepackage{titling}

\setlength{\parindent}{0pt}
\pagenumbering{gobble}

\newcommand{\cvtitle}[2]{
    \pretitle{\begin{flushleft}}
    \title{\huge \textbf{#1} \hfill \normalsize #2}
    \author{}
    \date{}
    \posttitle{\par\end{flushleft}\hrule\vspace{-0.75in}}
}

\newcommand{\experience}[2]{
    \subsection*{#1 \hfill \normalfont{#2}}
}

\newcommand{\contactinfo}[3]{
    \faIcon{#1} \href{#2://#3}{#3}
}

\newcommand{\otherinfo}[3]{
    \faIcon{#1} {#3}
}

\newcommand{\skillclass}[2]{
    \textbf{#1:} & #2
}

\begin{document}

\cvtitle{Bernardo Barzotto Zomer}{Engenheiro de Observabilidade}
\setlength{\droptitle}{-0.75in}
\maketitle

\begin{tabular}{ l l }
    \contactinfo{envelope-square}{mailto}{bernardobarzottoz@gmail.com} &
    \contactinfo{phone-square-alt}{tel}{+55-51-99652-7012} \\
    \contactinfo{linkedin}{https}{linkedin.com/in/bernardo-b-zomer} &
    \contactinfo{github-square}{https}{github.com/Bernardozomer} \\
    \contactinfo{pen-square}{https}{lattes.cnpq.br/0916138424420417} & 
    \otherinfo{map-marker-alt}{}{Porto Alegre, RS, Brasil} \\
\end{tabular}

\section*{Competências}

\begin{tabular}{ l l }
    \skillclass{Ferramentas}{Grafana, Docker, Terraform, Ansible, Git, GitHub;} \\
    \skillclass{Bancos de dados}{Prometheus, InfluxDB, SQL Server, MySQL, PostgreSQL, MongoDB;} \\
    \skillclass{Linguagens}{Python, Rust, Java, C, C++, Shell, SQL;} \\
    \skillclass{Idiomas}{Inglês fluente (apto a leitura, escrita e conversas técnicas), Português nativo.} \\
\end{tabular}

\section*{Experiência}

\experience{Dell Technologies}{Engenheiro de Observabilidade}

\emph{08/2023 a 02/2025 (Engenheiro de Software I), 02/2025 até o presente (Engenheiro de Software II).}

Desenvolvimento de dashboards no Grafana para aumentar a proatividade de times de SRE. Correlação de métricas (Prometheus, InfluxDB) com databases SQL. Desenvolvimento de testes sintéticos e alertas. Colaboração com times de engenharia de dados para expandir a cobertura do monitoramento. Desenvolvimento de soluções para incidentes P1 e avaliação das ferramentas oferecidas com base em incidentes passados. Automação de processos de infraestrutura F5 com Python e Flask. 

\experience{PUCRS/Dell IT Infra Residency}{Estágio}

\emph{04/2023 a 08/2023.}

Residência em infraestrutura de TI através de um treinamento imersivo e capacitação técnica com ênfase no estudo, planejamento e desenvolvimento de processos de automação. Realização de um projeto de IaC com provisionamento de infraestrutura na AWS por meio de Terraform e Ansible, com bancos de dados NoSQL MongoDB com redundância (replicação e sharding) e streaming de dados de vídeo. Monitoramento dos bancos de dados com Grafana e Prometheus.

\experience{Grupo de Modelagem de Aplicações Paralelas (GMAP)}{Bolsa de Iniciação Científica}

\emph{10/2022 a 04/2023.}

Atuação como pesquisador na área de programação paralela e distribuída, com foco em computação de alto desempenho, avaliação de algoritmos, otimização projeto de soluções e ciência de dados. Paralelização do NAS Parallel Benchmarks (NPB) com OpenMP e OneTBB (C++), conversão do kernel EP do NPB para Rust e paralelização de aplicações de streaming.

\section*{Certificados}

\begin{itemize}
	\item \href{https://www.credly.com/badges/69bb404e-ae99-4a93-96b1-228fa98d0859/public_url}{AWS Academy Graduate - AWS Academy Introduction to Cloud Semester 1}
\end{itemize}

\section*{Formação}

Graduação em \textbf{Ciência da Computação} na Pontifícia Universidade Católica
do Rio Grande do Sul (\textbf{PUCRS}). Início em 2021/1 e previsão de conclusão
em 2025/1.

\end{document}
